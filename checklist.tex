% ------------------------------------------------------------------
% Setup / Dokumententyp
% ------------------------------------------------------------------
\documentclass[11pt,oneside,ngerman]{book}
% ------------------------------------------------------------------
% Setup / Packages allgemein
% ------------------------------------------------------------------
\usepackage{helvet}
\renewcommand{\familydefault}{\sfdefault}
\usepackage[T1]{fontenc}
\usepackage[latin9]{inputenc}
\usepackage[a4paper]{geometry}
\geometry{tmargin=2.5cm,bmargin=2cm,lmargin=2.5cm,rmargin=2.5cm}
\usepackage{setspace}
\onehalfspacing
\usepackage[ngerman]{babel}
\setlength{\parindent}{0pt}
\newcommand{\HRule}{\rule{\linewidth}{0.5mm}}
\usepackage[geometry]{ifsym}
\usepackage{amsmath}
\usepackage{amssymb}
% ------------------------------------------------------------------
% Setup / Packages PDF-Generierung
% ------------------------------------------------------------------
\usepackage{hyperref}
\hypersetup{
    unicode=true,           		% non-Latin characters in Acrobat's bookmarks
    pdftoolbar=true,        		% show Acrobat's toolbar?
    pdfmenubar=true,        		% show Acrobat's menu?
    pdffitwindow=true,	     		% window fit to page when opened
    pdfstartview={FitH},   		% fits the width of the page to the window
    pdftitle={?},			% title
    pdfauthor={G�nther Eberl},  	% author
    pdfsubject={?},			% subject of the document
    pdfcreator={gVim, Miktex 2.9.4250}, % creator of the document
    pdfproducer={PdfLaTeX},		% producer of the document
    pdfkeywords={?}			% list of keywords
    pdfnewwindow=true,      		% links in new window
    colorlinks=true,        		% false: boxed links; true: colored links
    linkcolor=black,        		% color of internal links
    citecolor=black,        		% color of links to bibliography
    filecolor=red,      		% color of file links
    urlcolor=blue,         		% color of external links
    bookmarksopen=true			% open the bookmarks panel when Acrobat starts
}
% ------------------------------------------------------------------
% Dokument beginnen
% ------------------------------------------------------------------
\begin{document}

\pagestyle{empty}

% Angekreuztes K�stchen:
%\item [\rlap{\Cross}\BigSquare]


\paragraph{Stil}

\begin{itemize}
	\item [\BigSquare] \textbf{Hochsprache:} Nutzen. Genauigkeit. Technische/sachliche Formulierung.
	\item [\BigSquare] \textbf{K�rzen:} Wo m�glich. Keine Redundanz. So einfach und klar wie m�glich. Satzl�nge maximal 2-3 Zeilen.
	\item [\BigSquare] \textbf{Fachw�rter:} Bewusst einsetzen, auf ein St�ck pro Satz beschr�nken.
	\item [\BigSquare] \textbf{Nullw�rter:} Entfernen. Verhindern Pr�zission.
		\begin{itemize}
			\item [\FilledSmallCircle] Schl�sselw�rter: Wirklich, nun, ja, gar, so, ungef�hr, oft, obwohl, als wenn, beinahe, bis hin zu, sehr, je, manchmal, nur.
		\end{itemize}
	\item [\BigSquare] \textbf{Relativ- und Schachtels�tze:} Aufl�sen durch Zweiteilen. Maximal ein Komma pro Satz.
	\item [\BigSquare] \textbf{Anglizismen:} Wenn irgendwie m�glich vermeiden.  
		\begin{itemize}
			\item [\FilledSmallCircle] Nur zugelassen, wenn der Begriff ein Fachwort darstellt und keine �quivalente �bersetzung existiert (z.B. Compiler \(\neq\) �bersetzer).
			\item [\FilledSmallCircle] Verben sinngem�� �bersetzen, Verwendung von 'geparsed' oder 'gehashed' ist nicht zul�ssig.
		\end{itemize}
	\item [\BigSquare] \textbf{Abk�rzungen:} Neutrale ausschreiben, au�er wenn griechischem Ursprungs. Fachbezogene bei erster Verwendung definieren.
		\begin{itemize}
			\item [\FilledSmallCircle] Verboten: bzw., z.B., usw.
			\item [\FilledSmallCircle] Erlaubt: etc., ca.
		\end{itemize}
	\item [\BigSquare] \textbf{Klare Begriffe:} Nicht nur 'System', sondern 'Ticketsystem', 'Betriebssystem' oder 'Server'. Pr�zise sagen was gemeint ist.
	\item [\BigSquare] \textbf{Personifizierungen:} Nicht verwenden. 'F�r OBD-Fehlerverwaltung ist ...' \begin{math}\rightarrow\end{math} 'Im Bereich OBD-Fehlerverwaltung ist ...'
	\item [\BigSquare] \textbf{Passiv:} Form nicht nutzen. 'Ein Fehler wird ausgel�st.' \begin{math}\rightarrow\end{math} 'Die Verschlechterung der Abgasqualit�t l�st einen Fehler aus.'
		\begin{itemize}
			\item [\FilledSmallCircle] Bringt Information rein \textit{wer} etwas macht und \textit{warum}. Pr�ziser.
			\item [\FilledSmallCircle] Schl�sselw�rter: wird, wurde, geworden, gemacht.
		\end{itemize}

	\item [\BigSquare] \textbf{Adjektive:} Entfernen. Bringt Wissenschaftlichkeit, K�rze, Trockenheit, Knackigkeit. Verhindert Blumigkeit, Schw�lstigkeit, Werbungs-Stil. Lebendigkeit entsteht �ber Verben.
		\begin{itemize}
			\item [\FilledSmallCircle] Verboten: 'Der \textit{tolle} Editor', 'Der \textit{gro�artige} Compiler'.
			\item [\FilledSmallCircle] Stattdessen: Vorz�ge nennen, konkret sein.
		\end{itemize}
\pagebreak
	\item [\BigSquare] \textbf{Substantivierte Verben:} Vermeiden, Satz umformulieren.
		\begin{itemize}
			\item [\FilledSmallCircle] Erkennbar durch Endung '-ung'
			\item [\FilledSmallCircle] 'Durch die \textit{Compilierung} wird die Software zu einem Binary.' \begin{math}\rightarrow\end{math} 'Der \textit{gcc} (SUBJEKT) \textit{kompiliert} (VERB) \textit{die Sourcen} (OBJEKT) zu einem Binary (ZIEL).'
		\end{itemize}
	\item [\BigSquare] \textbf{Unpers�nlich schreiben:} Kein 'ich'/'man'/'wir'/'uns', ausser im evtl. vorhandenen Vorwort.
		\item [\BigSquare] \textbf{3-W�rter-Regel:} Bis aus drei W�rtern zusammengesetzte Substantive werden ohne Bindestrich geschrieben, ab vier wird sinnvoll geteilt.
			\item [\FilledSmallCircle] Verboten: 'Ticket-Ersteller', 'Ticket-Bearbeiter', 'Klo-Deckel-Halter'
			\item [\FilledSmallCircle] Stattdessen: 'Ticketersteller', 'Ticketbearbeiter', 'Klodeckelhalter'.
\end{itemize}

\paragraph{Quellen}

\begin{itemize}
	\item [\BigSquare] \textbf{Format anpassen}: Drei Buchstaben + Jahr + a/b/c.
	\item [\BigSquare] \textbf{Speichern}: *.mht, notfalls *.png.
	\item [\BigSquare] \textbf{Abbildungen}:
			\begin{itemize}
			\item [\FilledSmallCircle] Bevorzugt selbst neu zeichnen, dabei irrelevantes rauslassen, Basis-Quelle angeben ('[Src99a], ver�nderte Darstellung').
			\item [\FilledSmallCircle] Nicht selbst gezeichnete: Quelle angeben.
			\item [\FilledSmallCircle] M�ssen mindestens ein Mal im Text referenzeirt sein. Wiederholungen vermeiden ('siehe', 'vergleiche', 'nach').
		\end{itemize}
	\item [\BigSquare] \textbf{Tabellen}: M�ssen mindestens ein Mal im Text referenzeirt sein. Wiederholungen vermeiden ('siehe', 'vergleiche', 'nach').
\end{itemize}

\paragraph{Layout}

\begin{itemize}
	\item [\BigSquare] \textbf{Silbentrennung:} Ausschalten. Wenn Blocksatz als Folge davon zu kurz/lang f�r die Zeile wird den Satz umstellen.
	\item [\BigSquare] \textbf{Seitenumbr�che:} Sinnvoll legen. Nicht mitten in einer Liste.
	\item [\BigSquare] \textbf{Elemente trennen:} Abbildung nicht direkt folgend auf Tabelle und andersrum.
\end{itemize}

\end{document}\grid
\grid

